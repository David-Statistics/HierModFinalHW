\documentclass[12pt, a4paper]{article}
\usepackage[letterpaper, portrait, margin=1 in]{geometry}
\usepackage{amsmath}
\usepackage{amsfonts}
\usepackage{xcolor}
\usepackage{amssymb}
\usepackage{setspace}
\newcommand*\diff{\mathop{}\!\mathrm{d}}
\newcommand*\Diff[1]{\mathop{}\!\mathrm{d^#1}}
\usepackage{graphicx} 
\begin{document}
\doublespacing

\section{Background} 

$\indent$ Important drivers of mammalian population dynamics include juvenile survival and birth rate (cite). The number of actively reproducing females in a population reflects these two factors. We investigated the trend over time in the abundance of parturient females in a population of California sea lions on San Miguel Island. Parturient females were defined for a given year as those that had given birth that year. We were interested in how parturient female abundance changes over time in response to variables such as el Nino and age of the individual. 

The data are from a long-term capture-resight study conducted by Melin et al (2012) on San Miguel Island. Between 1987 and 2001, researchers branded 3945 healthy 4-5 month old female sea lion pups. The researchers took observations of the population during the summers in the years 1991-2008, recording each time a branded sea lion was resighted. They recorded the sea lion's brand number, date and location of resight, and indicated if the sea lion was seen with her own pup or not. 

We analyzed a subset of the data consisting of a cohort of female sea lion pups branded in 1991 and resighted between 1992 and 2008. We fit a hierarchical abundance model to estimate the number of individuals from the 1991 cohort that gave birth in each resight year. Our model could be generalized further to estimate the total number of parturient females in the population in a given year. These estimates provide information about population size, growth, and health, and are therefore useful for wildlife management, conservation, and research purposes. 

\section{Build Model}


\end{document}
