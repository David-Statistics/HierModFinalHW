\documentclass[12pt, a4paper]{article}
\usepackage[letterpaper, portrait, margin=1 in]{geometry}
\usepackage{amsmath}
\usepackage{amsfonts}
\usepackage{xcolor}
\usepackage{cite}
\usepackage{amssymb}
\usepackage{setspace}
\newcommand*\diff{\mathop{}\!\mathrm{d}}
\newcommand*\Diff[1]{\mathop{}\!\mathrm{d^#1}}
\usepackage{graphicx} 

\begin{document}
\doublespacing

\section{Background} 

$\indent$ Survival and reproduction are key elements of a successful population. Lifetimes can provide information about population growth and size, and recruitment times can inform population growth rates. Recruitment times are defined as the time it takes for an individual to become parous, or to reproduce for the first time. Both survival rates and  recruitment may be affected by environmental factors and can influence the health and stability of a population. We investigated the lifetime and recruitment time in a population of California sea lions. Estimates of these population parameters can be used by researchers to assess population trends over time for management, conservation, and scientific purposes.  

The data are from a long-term capture-resight study conducted by Melin et al on San Miguel Island \cite{Melin}. Between 1987 and 2001, researchers branded 3945 healthy 4-5 month old female sea lion pups. The researchers took observations of the population during the summers in the years 1991-2008, recording each time a branded sea lion was resighted. They recorded the sea lion's brand number, date and location of resight, and indicated if the sea lion was seen with her own pup or not. This long-term capture-resight study provides useful data for estimating various population parameters, specifically time until recruitment and death. 

\section{Build Model}

We analyzed a subset of the data consisting of a cohort of 259 female sea lion pups branded in 1991 and resighted between 1992 and 2008. To estimate time until recruitment, $r_i$, and time until death, $d_i$, for each individual in the 1991 cohort, we fit the following hierarchical model. \\
\\
For $i = 1, \ldots n$, $t = 1, \ldots, T$, and $r_i \leq d_i$ for all $i$, 

\begin{eqnarray*}
y_{it} & \sim & \begin{cases} 0 & , t \leq r_i \\
\text{Bern}(p) & , t > r_i \end{cases} \\
z_{it} & \sim & \begin{cases} 0 & , t > d_i \\
\text{Bern}(\psi) & , t \leq d_i \end{cases} \\
\log(r_i) & \sim & \text{N}(\mu_r, \sigma_r^2) \\
d_i & \sim & \exp(\lambda) \\
\mu_r & \sim & \text{N}(\mu_0, \sigma_0^2) \\
\sigma_r^2 & \sim & \text{IG}(\alpha_{\sigma}, \beta_{\sigma}) \\
\lambda & \sim & \text{Gamma}(\alpha_{\lambda}, \beta_{\lambda}) \\
p & \sim & \text{Beta}(\alpha_p, \beta_p) \\
\psi & \sim & \text{Beta}(\alpha_{\psi}, \beta_{\psi} )\\
\end{eqnarray*}
where $y_{it}$ indicates if individual $i$ was observed with a pup in year $t$, $z_{it}$ indicates if individual $i$ was observed in year $t$, $p$ is the probability of detecting an individual with her pup given that she was alive and parous, $\psi$ is the probability of detecting an individual given that she was alive, $r_i$ is the recruitment time for individual $i$, and $d_i$ is the lifetime for individual $i$. 

\section{Prior Specification}
$\indent$ The parameter $\lambda$ reflects the inverse of the average lifetime for California sea lions. Previous research estimates sea lion life expectancy to be around 20-30 years \cite{NOAA}. We let $\alpha_{\lambda} = 4$ and $\beta_{\lambda} = 100$, corresponding to E$(\lambda) = .04$ and Var$(\lambda) = .004$. Thus the expected lifetime is 25 years, with most of the mass falling near the mean, but allowing for individual lifetimes to vary substantially. 

We chose the prior hyperparameters for $\mu_r$ and $\sigma_r^2$ based off recent research on South American sea lions. In a study by Grandi et al, researchers estimated that female South American sea lions reach sexual maturity around 4.8 $\pm$ 0.5 years \cite{Grandi}. We believed California sea lions would mature at a similar age and therefore let $\mu_0 = 1.6$ and $\sigma_0^2 = .0625$, roughly corresponding to an average recruitment time of 5 years and allowing for moderate variation around this mean. We let $\alpha_{\sigma} = 3$ and $\beta_{\sigma} = 1$ so the expected variation in $\log(r_i)$ is E$[\sigma_r^2] = 0.5$ and Var$[\sigma_r^2] = 0.25$. This allows individual recruitment times to have reasonable variability around the mean. 

We set the prior hyperparameters for $p$ as $\alpha_p = 1.5$ and $\beta_p = 4.5$. Then E$[p] = .25$, with a fair amount of uncertainty around this mean. This reflects the idea that we do not have a great deal of information regarding $p$, but we believe $p$ may be relatively low since not all fertile sea lions reproduce and mothers do not spend all of their time with their pups. We set the hyperparameters for $\psi$ as $\alpha_{\psi} = 2$ and $\beta_{\psi} = 2$. This corresponds to E$[\psi] = .5$, with most mass centered around .5 and much less mass near the tails at 0 and 1. This reflects the idea that we believe $\psi$ is likely to be higher than $p$, but we still have little information regarding this parameter.  

\section{Posterior Distribution}

\begin{eqnarray*}
[\log (\mathbf{r}), \mu_r, p, \mathbf{d}, \lambda, \psi | \mathbf{z}, \mathbf{y} ] & \propto & \prod_{i=1}^n \prod_{t=1} ^T  1\{r_i \leq d_i \} [y_{it}|p,r_i]^{1\{ t > r_i \}} 1^{1\{ t \leq r_i \}} [z_{it}|\psi, d_i]^{1 \{ t \leq d_i\} } 1^{1 \{ t > d_i\}} \\
  & \times & [\log(r_i)|\mu_r][d_i|\lambda]  [\mu_r][\lambda][p][\psi]
\end{eqnarray*}


\section{Algorithm}

We derived the full conditionals for $p$, $\psi$, $\lambda$, and $\sigma_r^2$ (Appendix) and updated those parameters with a Gibbs sampler. We updated $\mathbf{d}, \mathbf{r}$, and $\mu_r$ with Metropolis-Hastings. We ran the MCMC algorithm for 100,000 iterations with a burn-in period of 50,000. Resulting trace plots showed thorough mixing and convergence of parameters ?!?!?!. 

\section{Results}

Model is closed in space but the data are not. $d_i$ ideally is lifetime, but practically is the time of leaving the system. 

How do we want to present results. 

Next steps: use within year data to inform detection.

\section{Inference}



\section{Appendix}

\subsection{Full Conditionals} 

Full conditional distributions were derived for the parameters $p$, $\psi$, $\lambda$, and $\sigma_r^2$:
\begin{eqnarray*}
p | \cdot & \sim & \text{Beta} \left( \alpha_p + \sum_{i=1}^n \sum_{t=1}^T 1 \{ r_i \leq t \leq d_i \} y_{it}, \beta_p + \sum_{i=1}^n \sum_{t=1}^T 1\{ r_i \leq t \leq d_i \} (1 - y_{it}) \right)  \\
\lambda | \cdot & \sim & \text{Gamma} \left( \alpha_{\lambda} + n, \beta_{\lambda} + \sum_{i=1}^n d_i \right) \\
\psi | \cdot & \sim & \text{Beta}\left( \alpha_{\psi} + \sum_{i=1}^n \sum_{t=1}^T 1\{ t \leq d_i\} z_{it}, \beta_{\psi} + \sum_{i=1}^n \sum_{t=1}^T 1\{ t \leq d_i \} (1 - z_{it}) \right) \\
\sigma_{r}^2 | \cdot & \sim & \text{IG} \left( \alpha_{\sigma} + \frac{n}{2}, \beta_{\sigma} + \frac{1}{2} \sum_{i=1}^n ( \log(r_i) - \mu_r)^2 \right)
\end{eqnarray*}
The parameters $\mathbf{d}, \mathbf{r}$, and $\mu_r$, did not have recognizable full conditionals so we updated them with Metropolis-Hastings.  

\begin{thebibliography}{99}

\bibitem{NOAA}
NOAA: California Sea Lion, \\
\text{http://www.nmfs.noaa.gov/pr/species/mammals/sealions/california-sea-lion.html}

\bibitem{Grandi}
Grandi, M.F., Dans, S.L., Garcia, N.A., Crespo, E.A. 2010. \emph{Growth and age at sexual maturity of South American sea lions}. Mammalian Biology, 75(5):427-436.   

\bibitem{Melin}
Melin, Sharon R.; DeLong,Robert L.; Orr, Anthony J.; Laake, Jeffrey L.; Harris, Jeffrey D.; NOAA NMFS Alaska Fisheries Science Center, Marine Mammal Laboratory (2016). Survival and natality rate observations of California sea lions at San Miguel Island, California conducted by Alaska Fisheries Science Center, National Marine Mammal Laboratory from 1987-09-20 to 2014-09-25 (NCEI Accession 0145167). Version 2.2. NOAA National Centers for Environmental Information. Dataset. [Nov. 19, 2017]

\end{thebibliography}

\end{document}
