\documentclass[12pt, a4paper]{article}
\usepackage[letterpaper, portrait, margin=1 in]{geometry}
\usepackage{amsmath}
\usepackage{amsfonts}
\usepackage{xcolor}
\usepackage{natbib}
\usepackage{amssymb}
\usepackage{setspace}
\usepackage{bm, bbm}
\bibliographystyle{agsm}
\newcommand*\diff{\mathop{}\!\mathrm{d}}
\newcommand*\Diff[1]{\mathop{}\!\mathrm{d^#1}}
\usepackage{graphicx} 

\newcommand{\citeay}[1]{\citeauthor{#1} \citeyear{#1}}

\begin{document}
\doublespacing

\section{Background} 

$\indent$ Survival and reproduction are key elements of a successful population. Lifetimes provide information about population growth and size, and recruitment times inform population growth rates. Recruitment time is defined as the time it takes for an individual to become parous, or to reproduce for the first time. Both survival rates and  recruitment influence the health and stability of a population and may be affected by environmental factors. We investigated lifetime and recruitment time in a population of California sea lions. Estimates of these population parameters can be used by researchers to assess population trends over time for management, conservation, and scientific purposes.  

The data are from a long-term mark-resight study conducted by \citet{Melin} on San Miguel Island. Between 1987 and 2001, researchers branded 3945 healthy 4-5 month old female sea lion pups. The researchers took observations of the population during the summers in the years 1991-2008, recording each time a branded sea lion was resighted. They recorded the sea lion's brand number, date and location of resight, and indicated if the sea lion was seen with her own pup or not. This long-term capture-resight study provides useful data for estimating various population parameters, specifically time until recruitment and death. We analyzed a subset of these data consisting of a cohort of 259 female sea lion pups branded in 1991 and resighted between 1992 and 2008.

\section{Build Model}

We propose the following hierarchical Bayesian model for lifetime, $\bm{d}$, and recruitment time, $\bm{r}$ in the San Miguel Island population of California sea lions.\\

\begin{eqnarray*}
y_{it} & \sim & \begin{cases} 0 & , t \leq r_i \\
\text{Bern}(p) & , t > r_i \end{cases} \qquad i = 1,\ldots,n\text{; } t = 1,\ldots,T \\
z_{it} & \sim & \begin{cases} 0 & , t > d_i \\
\text{Bern}(\psi) & , t \leq d_i \end{cases} \\
f &=& (.8)\left(\frac{1}{8} \exp(-\frac{x}{8})\right) + (.2)\left(\frac{1}{3\sqrt{2\pi}} \exp \left(-\frac{1}{18} (x - 25)^2 \right)\right) \\
t_i^* & = & \max(\{t : y_{it} = 1\}) \\
d_i & \propto & f (\mathbbm{1}_{(d_i > t_i^*)}) (\mathbbm{1}_{(d_i > r_i)})  \\
\log(r_i) & \sim & \text{N}(\mu_r, \sigma_r^2) \\
\mu_r & \sim & \text{N}(\mu_0, \sigma_0^2) \\
\sigma_r^2 & \sim & \text{IG}(\alpha_{\sigma}, \beta_{\sigma}) \\
p & \sim & \text{Beta}(\alpha_p, \beta_p) \\
\psi & \sim & \text{Beta}(\alpha_{\psi}, \beta_{\psi} )\\
\end{eqnarray*}
In the data model, $y_{it}$ indicates if individual $i$ was observed with a pup in year $t$ and $z_{it}$ indicates if individual $i$ was observed in year $t$. The constant $t_i^*$ is the last time at which individual $i$ was observed in the study.
The probability of an individual being detected with her pup assuming she was alive, at the site, and parous is denoted $p$. Therefore, $p$ includes both the probability that a parous female had a pup and that she was observed alongside her pup. The probability of detecting an individual given that she was alive and at the site is given by $\psi$.

\section{Prior Specification}

The prior distribution for $d_i$ is proportional to an 80/20 mixture of an exponential distribution with mean 8 and a normal distribution with mean 25 and variance 9. Previous research studying age-specific survival rates of California sea lions suggests the probability of a young female sea lion surviving the first year of life ranges from .558 to .998. After the first year, the probability of survival from year $t$ to year $t+1$ is roughly 0.9 \citep{Hernandez}. Our prior specification for $d_i$ emulates these findings. The prior probability of survival to year 1 is 0.905, to year 4 is .685, and to year 10 is .43. This reflects lower survival probabilities for young and old individuals and higher survival probabilities for middle-aged individuals. We include in our prior distribution on $d_i$ the fact that if an individual $i$ was last seen at time $t_i^*$, then $d_i$ must be greater than $t_i^*$. 

We chose the prior hyperparameters for $\mu_r$ and $\sigma_r^2$ based off recent research on South American sea lions. \citet{Grandi} estimated that female South American sea lions reach sexual maturity around 4.8 $\pm$ 0.5 years . We believed California sea lions would mature at a similar age and therefore let $\mu_0 = 1.6$ and $\sigma_0^2 = .0625$, roughly corresponding to an average recruitment time of 5 years and allowing for moderate variation around this mean. We let $\alpha_{\sigma} = 3$ and $\beta_{\sigma} = 1$ so the expected variation in $\log(r_i)$ is E$[\sigma_r^2] = 0.5$ and Var$[\sigma_r^2] = 0.25$. This allows individual recruitment times to have reasonable variability around the mean. 

We set the prior hyperparameters for $p$ as $\alpha_p = 1.5$ and $\beta_p = 4.5$. Then E$[p] = .25$, with a fair amount of uncertainty around this mean. We do not have a great deal of information regarding $p$, but we believe $p$ may be relatively low since not all fertile sea lions reproduce and mothers may be spotted without their pups. We set the hyperparameters for $\psi$ as $\alpha_{\psi} = 2$ and $\beta_{\psi} = 2$. This corresponds to E$[\psi] = .5$, with most mass centered around .5 and much less mass near 0 and 1. We believe $\psi$ is likely to be higher than $p$ and unlikely to be either very close to 0 or very close to 1. The former case would imply that the individuals rarely return to the island or are very unlikely to be detected. The latter would imply that animals almost always return to the island and they are almost always detected.

\section{Posterior Distribution}

Using the model given above, the posterior distribution for the parameters is,

\begin{eqnarray*}
[\log (\mathbf{r}), \mu_r, p, \mathbf{d}, \lambda, \psi | \mathbf{z}, \mathbf{y} ] & \propto & \prod_{i=1}^n \prod_{t=1} ^T  1\{r_i \leq d_i \} [y_{it}|p,r_i]^{1\{ t > r_i \}} 1^{1\{ t \leq r_i \}} [z_{it}|\psi, d_i]^{1 \{ t \leq d_i\} } 1^{1 \{ t > d_i\}} \\
  & \times & [\log(r_i)|\mu_r][d_i|\lambda]  [\mu_r][\lambda][p][\psi].
\end{eqnarray*}


\section{Algorithm}

We derived the full conditionals for $p$, $\psi$, $\mu_r$, and $\sigma_r^2$ (Appendix) and updated those parameters with a Gibbs sampler \citep{Geman1984}. We used Metropolis-Hastings updates for $\mathbf{d} and \mathbf{r}$ (\citeay{Metropolis1953}, \citeay{Hastings1970}). We fit the above model by running the Markov Chain Monte Carlo algorithm for 50,000 iterations with a burn-in period of 5,000. Resulting trace plots showed thorough mixing and convergence of parameters. 

\section{Results}

\begin{table}[ht]
\centering
\begin{tabular}{rrrr}
  \hline
   & Posterior Mean & 95\% CI Lower & 95\% CI Upper \\ 
  \hline
  p & 0.12 & 0.10 & 0.13 \\ 
  psi & 0.27 & 0.25 & 0.30 \\ 
   \hline
\end{tabular}
\end{table}

\begin{figure}[h]
\centering
\includegraphics[width = \textwidth]{Posterior_d.png}
\caption{Posterior and prior distributions on $d$ for individuals which we last saw in certain years.}
\end{figure}

\begin{figure}[h]
\centering
\includegraphics[width = .4\textwidth]{Posterior_r.png}
\caption{Posterior and prior distributions on $r$.}
\end{figure}

\section{Inference}

Ideally, the parameter $d_i$ represents the lifetime of individual $i$, but practically it represents the time the individual left the population, which could be due to death or emigration. Future research could address this issue by also modeling the probability of emigration using telemetry data. 

Our model predicts varying times for $d_i$ based on the last time the individual $i$ was observed in the study period. For individuals not observed after the first few years, the estimated time in the population tended to relatively low, with most mass below 10 years. Individuals only observed in the first few years of the study, may have simply never been detected again, but our posterior distribution reflects substantial probability that the individual actually left the system. For individuals observed later into the study, the estimated time in the system tended to be dominated by the prior distribution, and the posterior mass was concentrated around 25 years. Given that the study period only continued for 17 years after marking the pups, the data do not provide information on the lifespan of individuals beyond that point. Thus, the prior and posterior distributions for these individuals should, intuitively, be very similar.

If the system were spatially closed, the parameter $\psi$ denotes the probability of detecting an individual assuming she was alive. However, this was not a closed system. Thus, $\psi$ denotes the probability of an individual returning to site and actually detecting the individual. Our posterior mean for $\psi$ is $0.27~(95\% CI: [.25, .30])$. 

The parameter $p$ is a complex parameter in this model. It is the probability of a parous female having a pup and having the researchers observing her with the pup given that the individual was alive and at the site. The posterior mean for $p$ is $0.12~(95\% CI: [.10, 13])$. Relatively low detection probability is not surprising given there are many reasons that an individual with a pup may not be detected with her pup. Not long after birth, sea lion mothers tend to leave their pups to hunt, so it is possible to detect a sea lion without her pup even if she had a pup in that year. Furthermore, not all parous females have pups ever year, and $p$ should account for this. 

Our posterior distributions for $r_i$, the time until individual $i$ became parous, were not substantially different across individuals that were last sighted at different times.  The posterior means for $r_i$ were typically between 0.243 and 2.004 years. Based on prior information that sea lions typically become parous around 4-5 years, our model is under-estimating this parameter. This is likely due to the low probability of detecting an individual with her pup and the constraint that $r_i < d_i$. This constraint might be relaxed in the future, since an individual may in fact die or leave the population before she ever becomes parous. In this case we would have $d_i < r_i$, where $r_i$ is the theoretical time at which the individual would have become parous if she had lived or remained in the population long enough. 

\section{Further Research}

One obvious extension to the model and results discussed here is accounting for all of the pups on which we have data. Previously, we mentioned that there is reason to believe that survival and recruitment times may depend on covariates (e.g., El-Nino Southern Oscillation). These effects could be taken into account when the data include pups born in different years. 

The data from this research study also included repeated sightings within a year; utilizing these within-year data would improve the intepretability of parameters in the model. We would be able to separate $p$ into two parameters describing the probability of detecting the pup and probability of the individual having a pup. In addition, we could separate $\psi$ into a probability of detection and probability of the individual returning to the same site. However, using this approach may increase the complexity of the model by a large margin as the level of effort was not constant throughout the period. In addition, it is not clear what an "occassion" should be in a within-year context.


\section{Appendix}

\subsection{Full Conditionals}

Full conditional distributions were derived for the parameters $p$, $\psi$, $\lambda$, and $\sigma_r^2$:
\begin{eqnarray*}
p | \cdot & \sim & \text{Beta} \left( \alpha_p + \sum_{i=1}^n \sum_{t=1}^T 1 \{ r_i \leq t \leq d_i \} y_{it}, \beta_p + \sum_{i=1}^n \sum_{t=1}^T 1\{ r_i \leq t \leq d_i \} (1 - y_{it}) \right)  \\
\psi | \cdot & \sim & \text{Beta}\left( \alpha_{\psi} + \sum_{i=1}^n \sum_{t=1}^T 1\{ t \leq d_i\} z_{it}, \beta_{\psi} + \sum_{i=1}^n \sum_{t=1}^T 1\{ t \leq d_i \} (1 - z_{it}) \right) \\
\sigma_{r}^2 | \cdot & \sim & \text{IG} \left( \alpha_{\sigma} + \frac{n}{2}, \beta_{\sigma} + \frac{1}{2} \sum_{i=1}^n ( \log(r_i) - \mu_r)^2 \right)
\end{eqnarray*}
The parameters $\mathbf{d}, \mathbf{r}$, and $\mu_r$, did not have recognizable full conditionals so we updated them with Metropolis-Hastings.

% \begin{thebibliography}{99}
% 
% \bibitem{NOAA}
% NOAA: California Sea Lion, \\
% \text{http://www.nmfs.noaa.gov/pr/species/mammals/sealions/california-sea-lion.html}
% 
% \bibitem{Grandi}
% Grandi, M.F., Dans, S.L., Garcia, N.A., Crespo, E.A. 2010. \emph{Growth and age at sexual maturity of South American sea lions}. Mammalian Biology, 75(5):427-436.   
% 
% \bibitem{Melin}
% Melin, Sharon R.; DeLong,Robert L.; Orr, Anthony J.; Laake, Jeffrey L.; Harris, Jeffrey D.; NOAA NMFS Alaska Fisheries Science Center, Marine Mammal Laboratory (2016). Survival and natality rate observations of California sea lions at San Miguel Island, California conducted by Alaska Fisheries Science Center, National Marine Mammal Laboratory from 1987-09-20 to 2014-09-25 (NCEI Accession 0145167). Version 2.2. NOAA National Centers for Environmental Information. Dataset. [Nov. 19, 2017]
% 
% \bibitem{Hernandez}
% Hernandez-Camacho, Claudia J.; Aurioles-Gamboa, David; Laake, Jeffrey; Gerber, Leah R. 2008. Survival Rates of the California Sea Lion, Zalophus californianus, in Mexico, Journal of Mammalogy, 89(4): 1059–1066, https://doi.org/10.1644/07-MAMM-A-404.1
% 
% 
% \end{thebibliography}

\bibliography{finalHW}

\end{document}
